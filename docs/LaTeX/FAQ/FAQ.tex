\documentclass[12pt, a4paper]{article}
\usepackage[utf8]{inputenc}
\usepackage[T1]{fontenc}
\usepackage{lmodern}
\usepackage[brazil]{babel}
\usepackage[margin=2.5cm]{geometry} % Define as margens
\usepackage{hyperref} % Para URLs clicáveis
\usepackage{parskip} % Para espaçamento entre parágrafos em vez de indentação

% Configurações de Títulos de Seção
\usepackage{titlesec}
\titleformat{\section}
  {\normalfont\Large\bfseries} % Formato do título da seção
  {\thesection.}               % Numeração
  {1em}                        % Espaçamento
  {}                           % Código antes

\titleformat{\subsection}
  {\normalfont\large\bfseries} % Formato do título da subseção
  {\thesubsection}             % Numeração
  {1em}                        % Espaçamento
  {}                           % Código antes

% Título do Documento
\title{FAQ: Dúvidas Acadêmicas e Matrícula\\ \large Guia Rápido para Alunos}
\author{Secretaria Acadêmica (FASI/UFPA)}
\date{} % Remove a data

\begin{document}

\maketitle % Cria a página de título

% --- PERGUNTAS E RESPOSTAS ---

\section{Sou discente da Turma do PPC Antigo (2011, Limoeiro do ajuru, Oeiras do Pará e 2023) e como devo proceder sobre as disciplinas de Estágio I e II?}
Para os alunos do PPC Antigo, o processo de matrícula nas disciplinas de Estágio I e II será realizado pela secretaria acadêmica, conforme as diretrizes estabelecidas pela coordenação do curso. Siga as orientações abaixo:
\begin{enumerate}
    \item Estágio I
    \begin{itemize}
        \item A secretaria acadêmica realizará a matrícula na disciplina de Estágio I para todos os alunos que cumprirem os pré-requisitos necessários.
        \item O professor responsável pela disciplina deve orientar sobre Plano de Estágio I.
    \end{itemize}
    \item Estágio II
    \begin{itemize}
        \item Estágio propriamente dito. A secretaria acadêmica realizará a matrícula na disciplina de Estágio II para todos os alunos que cumprirem os pré-requisitos necessários.
        \item Para efetuar a matrícula, o aluno deverá apresentar o relatório de Estágio II.
    \end{itemize}   
    \item Em caso de dúvidas, o aluno deve entrar em contato com a secretaria acadêmica via e-mail ou presencialmente.
    \item Atenção: Todos os documentos devem estar devidamente assinados e podem ser encontrados na web page da FASI: \url{https://www.campuscameta.ufpa.br/index.php/sistemas-de-informacao?showall=&start=5}.
  \end{enumerate}


\section{QUAL DISCIPLINA DEVO ME MATRICULAR?}
Consulte o calendário de ofertas do curso disponível no site da Faculdade.

\section{COMO SOLICITAR AUMENTO DE CRÉDITOS?}
Entre em contato diretamente com Diretor ou Secretário por e-mail ou mensagem.

\section{O SISTEMA NÃO PERMITE QUE EU FAÇA DETERMINADAS DISCIPLINAS DEVIDO A CONFLITO DE HORÁRIO. O QUE DEVO FAZER?}
Se matricule em algumas disciplinas e deixe outras para depois. Lembre-se de que você não pode conquistar o mundo de uma vez. Seja paciente.

\section{REPROVEI EM UMA DISCIPLINA NO SEMESTRE PASSADO. O QUE DEVO FAZER?}
Estude e espere a próxima oferta da disciplina em um turno diferente do seu. Você poderá cursar essa disciplina novamente quando ela for ofertada (consulte o quadro de ofertas das disciplinas no site do curso).

\section{COMO FUNCIONAM AS DISCIPLINAS DE ESTÁGIO I E II?}
Essas disciplinas estão no quadro de ofertas do curso (no site), mas não são ofertadas diretamente no SIGAA. A direção, junto com a secretaria, realizará a matrícula durante o período definido no quadro de oferta.

\section{ME INSCREVI EM UMA DISCIPLINA POR ENGANO QUE NÃO FAZ PARTE DA MINHA GRADE. O QUE DEVO FAZER?}
Retorne à tela de matrícula e remova a disciplina indesejada. Atenção na próxima vez!

\section{RECEBI A MENSAGEM "NÃO É POSSÍVEL REALIZAR A MATRÍCULA PORQUE VOCÊ NÃO PREENCHEU A AVALIAÇÃO INSTITUCIONAL REFERENTE AO PERÍODO XXXX." O QUE DEVO FAZER?}
Você deve completar a Avaliação Institucional antes de realizar a matrícula. Vá na aba de "Ensino", selecione a opção "Avaliação Institucional" e preencha a avaliação.

\section{NENHUMA DISCIPLINA APARECE NO SIGAA. O QUE DEVO FAZER?}
Adicione as disciplinas manualmente. Siga estes passos: "Adicionar Turma" $\rightarrow$ "Adicionar disciplinas por código". Os códigos estão disponíveis na lista de ofertas da FASI.

\section{QUANDO SERÁ OFERTA DE UMA DETERMINADA DISCIPLINA?}
Consulte o calendário de ofertas do curso disponível no site da Faculdade.

\section{COMO INICIAR O PROCESSO DE OUTORGA DE GRAU (OU REGISTRO DE DIPLOMA)?}
\begin{enumerate}
    \item Entre no SIGAA
    \item Localize menu "Ensino".
    \item Localize opção "Solicitação Validação de Documentos para Registro de Diploma"
    \item Anexar: Diploma de Ensino Médio, Carteira de Identidade e Declaração de Quitação da Biblioteca
\end{enumerate}

\section{O ALUNO PODE FAZER DISCIPLINA (ATIVIDADE FLEXIBILIZADA) FORA DA UFPA? SE SIM, COMO SERÁ COMPUTADA NO SIGAA?}
\begin{itemize}
    \item Sim, o aluno pode cursar disciplinas em outras Instituições de Ensino Superior (IES).
    \item Para que as horas sejam validadas, a disciplina deve ser de uma área de conhecimento relevante para a sua formação. A Unidade Acadêmica da UFPA ou de outra IES que ofertar a disciplina será a responsável por computar a carga horária. A matrícula e o registro no SIGAA serão realizados de acordo com os procedimentos específicos definidos para o aproveitamento de estudos externos.
\end{itemize}

\section{O ALUNO PODE FAZER DISCIPLINA (ATIVIDADE FLEXIBILIZADA) NA UFPA, MAS FORA DO CAMPUS DE CAMETÁ? SE SIM, COMO ELE DEVE SOLICITAR NO OUTRO CAMPUS? VIA SIGAA?}
\begin{itemize}
    \item Sim, é possível cursar disciplinas em outros campi da UFPA.
    \item A matrícula nas atividades flexibilizadas dentro da UFPA será feita diretamente pelo SIGAA. Os cursos de graduação da UFPA podem disponibilizar vagas de livre acesso em suas disciplinas, que são destinadas a alunos de cursos com essa modalidade de flexibilização.
\end{itemize}

\section{O ALUNO PODE FAZER QUAIS TIPOS DE DISCIPLINAS (ATIVIDADE FLEXIBILIZADA)? EXISTE ALGUMA RESTRIÇÃO?}
\begin{itemize}
    \item O aluno pode escolher as disciplinas de acordo com seus interesses e preferências.
    \item A principal restrição é que a reserva de vagas para as atividades flexibilizadas não se aplica a disciplinas como TCC, estágios e práticas que sejam regulamentadas por normas específicas. Nesses casos, o aluno deve seguir as regras de seu curso.
\end{itemize}

\section{COMO POSSO ACESSAR ARTIGOS E PERIÓDICOS CIENTÍFICOS USANDO MEU LOGIN INSTITUCIONAL DA UFPA?}
O acesso ao acervo de periódicos da CAPES é um benefício para toda a comunidade acadêmica da UFPA. Para utilizá-lo, basta seguir este passo a passo para se autenticar com seu login institucional:

\subsection{Acesse o Portal de Periódicos da CAPES}
Acesse o site oficial pelo link: \url{https://www.periodicos.capes.gov.br/}

\subsection{Clique em ACESSO CAFE}
No menu superior da página, localize e clique na opção ACESSO CAFE.

\subsection{Selecione a UFPA}
Na tela da "Comunidade Acadêmica Federada", selecione "UFPA" na lista de instituições e clique no botão "Enviar".

\subsection{Faça o Login Institucional}
Você será redirecionado para a tela de autenticação da UFPA. Informe seu usuário e senha.
\begin*
    \textbf{Atenção:} O usuário deve ser o seu e-mail institucional completo (ex: \texttt{seulogin@ufpa.br} ou \texttt{seulogin@instituto.ufpa.br}).
\end*

\subsection{Acesso Concedido}
Pronto! Após o login, você terá acesso remoto a todas as bases de periódicos e artigos científicos assinados pela CAPES, como se estivesse conectado à rede da universidade.

\subsection{Como buscar por uma base específica (Exemplo: IEEE Xplore)?}
Após realizar o login, se você deseja acessar uma base de dados específica, como a do IEEE, siga estes passos:
\begin{enumerate}
    \item No menu do Portal, navegue até \textit{Acervo > Lista de bases e coleções}.
    \item Na barra de busca, digite “IEEE” e clique em “Enviar”.
    \item Nos resultados, clique sobre o nome da base (ex: IEEE Xplore Digital Library).
    \item Na página seguinte, clique no link para acessar a base.
\end{enumerate}

\end{document}