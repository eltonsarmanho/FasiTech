\documentclass{beamer}

\usepackage[utf8]{inputenc}
\usepackage[brazilian]{babel}
\usepackage{graphicx}
\usepackage{amsmath}
\usepackage{amsfonts}
\usepackage{amssymb}
\usepackage{hyperref}
\usepackage{tikz}
\usetikzlibrary{positioning, matrix}
\usetikzlibrary{calc}
\usetikzlibrary{shapes.geometric, arrows, arrows.meta, positioning}
\usetheme{Madrid}
\usepackage{fontawesome}
\usepackage{xcolor}
\usepackage{ccicons}

% Cores personalizadas (se quiser reaproveitar)
% Tema base
\usetheme{Madrid}

% === COR VERDE IGUAL AO LOGO DO EVENTO =====================
\definecolor{eventoGreen}{HTML}{00793A}
\usecolortheme[named=eventoGreen]{structure}

\setbeamercolor{title}{fg=white,bg=eventoGreen}
\setbeamercolor{frametitle}{fg=white,bg=eventoGreen}
\setbeamercolor{item}{fg=eventoGreen}
\setbeamercolor{block title}{fg=white,bg=eventoGreen}
\setbeamercolor{block body}{fg=black,bg=eventoGreen!5}

% Remover completamente o rodapé padrão
\setbeamertemplate{footline}{}
% ===========================================================


% Título dos Slides
\title{\large	{FasiTech: Automação Inteligente para a Gestão Acadêmica 4.0}}
\author[Elton Sarmanho]{%
  \Large{Elton Sarmanho\\[0.2cm]
  % Substitua "lattes-icon.png" pelo nome real do arquivo de ícone caso seja diferente
  \href{http://lattes.cnpq.br/6836143760585970}{\includegraphics[height=0.5cm]{lattes-icon.png}}\quad
  \href{https://github.com/eltonsarmanho}{\faGithub}\quad
  \href{https://www.linkedin.com/in/elton-sarmanho-836553185/}{\faLinkedinSquare}}%
}
\institute{UFPA – Faculdade de Sistemas de Informação - Campus Cametá}
\date{\ccbyncsa}

\begin{document}

% Capa
\begin{frame}
  % Se tiver logo do evento, você pode incluir assim:
   \begin{center}
     \includegraphics[scale=0.2]{logo_evento.png}
   \end{center}
    \titlepage

\end{frame}

% Sumário
\begin{frame}{Roteiro}
  \tableofcontents
\end{frame}

% =========================
\section{Contexto e Problema}

\begin{frame}{Contexto}
  \begin{itemize}
    \item Campus interiorizado, distante da capital e dos centros administrativos.
    \item Processos acadêmicos fortemente baseados em papel, e-mail e atendimento presencial.
    \item Dificuldade de acompanhar prazos, fluxos e status de solicitações.
    \item Sensação de morosidade e falta de transparência para estudantes e docentes.
  \end{itemize}
\end{frame}

\begin{frame}{Problemas Identificados}
  \begin{itemize}
    \item Dependência de deslocamento físico para entregar documentos e formular requerimentos.
    \item Retrabalho frequente na secretaria (documentos incompletos, formulários errados).
    \item Falta de registro estruturado para análise de dados e melhoria de processos.
    \item UFPA ainda distante de uma cultura de \textbf{Gestão 4.0} nos fluxos acadêmicos.
  \end{itemize}
\end{frame}

% =========================
\section{Antes do FasiTech}

\begin{frame}{Fluxo Antigo de Interação}
  \begin{itemize}
    \item Aluno preenche formulários em papel ou por e-mail, muitas vezes sem padrão.
    \item Secretaria recebe, confere manualmente e encaminha para coordenação/direção.
    \item Processos se acumulam em pilhas físicas ou caixas de e-mail.
    \item Dificuldade para saber “onde está” cada processo e quanto tempo leva.
  \end{itemize}
\end{frame}

\begin{frame}{Forma Antiga de Interação (Visual)}
  \begin{center}
    % Substituir pelo caminho correto da imagem de fluxo antigo
     \includegraphics[width=0.9\textwidth]{forma_antiga_1.png}
    %\emph{(Inserir aqui figura ilustrando o fluxo antigo de interação)}%
  \end{center}
\end{frame}
\begin{frame}{Forma Antiga de Interação (Visual)}
  \begin{center}
    % Substituir pelo caminho correto da imagem de fluxo antigo
     \includegraphics[width=0.9\textwidth]{forma_antiga_2.png}
    %\emph{(Inserir aqui figura ilustrando o fluxo antigo de interação)}%
  \end{center}
\end{frame}
% =========================
\section{O Projeto FasiTech}

\begin{frame}{O que é o FasiTech?}
  \begin{itemize}
    \item Plataforma web para centralizar formulários e fluxos acadêmicos da FASI/UFPA.
    \item Construída com tecnologias modernas:
      \begin{itemize}
        \item Frontend em \textbf{Streamlit}.
        \item Integrações com \textbf{Google Drive} e \textbf{Google Sheets}.
        \item Serviço de \textbf{e-mail} para notificações.
        \item Módulo de \textbf{IA/RAG} para apoio às decisões (Diretor Virtual).
      \end{itemize}
    \item Objetivo: transformar processos manuais em fluxos digitais, rastreáveis e inteligentes.
  \end{itemize}
\end{frame}

\begin{frame}{Arquitetura da Solução}
  \begin{center}
    % Substituir pelo caminho real da imagem de arquitetura
    \includegraphics[width=0.7\textwidth]{Arquitetura.png}
    %\emph{(Inserir aqui diagrama de arquitetura do FasiTech)}%
  \end{center}
\end{frame}

\begin{frame}{Automação de Processos (RPA)}
  \begin{itemize}
    \item Cada formulário gera registros estruturados em planilhas e pastas no Google Drive.
    \item Gatilhos para notificar secretaria, coordenação e docentes.
    \item Redução drástica de tarefas repetitivas e de risco de extravio de documentos.
  \end{itemize}
  \vfill
  \begin{center}
    \includegraphics[width=0.8\textwidth]{RPA.png}
    %\emph{(Inserir aqui figura ilustrando o fluxo automatizado com RPA)}%
  \end{center}
\end{frame}

% =========================
\section{Módulos e Funcionalidades}

\begin{frame}{Principais Módulos da Plataforma}
  \begin{itemize}
    \item \textbf{Formulários Acadêmicos}:
      \begin{itemize}
        \item ACC, Estágio, Projetos, Plano de Ensino, Requerimento de TCC, Formulário Social.
      \end{itemize}
    \item \textbf{Módulo Socioeconômico}:
      \begin{itemize}
        \item Coleta estruturada de dados para análise e planejamento de políticas de apoio.
      \end{itemize}
    \item \textbf{Ofertas de Disciplinas}:
      \begin{itemize}
        \item Organização das ofertas e acompanhamento das turmas.
      \end{itemize}
    \item \textbf{Diretor Virtual (IA/RAG)}:
      \begin{itemize}
        \item Responde perguntas sobre o PPC, ementas e normas internas.
      \end{itemize}
  \end{itemize}
\end{frame}

\begin{frame}{Diretor Virtual (RAG sobre PPC)}
  \begin{itemize}
    \item Agente de IA treinado sobre documentos oficiais do curso (PPC, ementas, resoluções).
    \item Arquitetura \textbf{RAG} (\textit{Retrieval-Augmented Generation}):
      \begin{itemize}
        \item Indexação de PDFs em um banco vetorial.
        \item Recuperação dos trechos mais relevantes para cada pergunta.
        \item Geração de respostas em linguagem natural com base nesses trechos.
      \end{itemize}
    \item Objetivo: apoiar decisões de coordenação e docentes, reduzindo consultas manuais a documentos extensos.
  \end{itemize}
\end{frame}

% =========================
\section{Impactos e Próximos Passos}

\begin{frame}{Benefícios Esperados}
  \begin{itemize}
    \item \textbf{Para estudantes}:
      \begin{itemize}
        \item Menos deslocamentos físicos e mais clareza sobre procedimentos.
        \item Acompanhamento mais transparente de solicitações.
      \end{itemize}
    \item \textbf{Para secretaria e coordenação}:
      \begin{itemize}
        \item Redução de retrabalho e de erros operacionais.
        \item Dados estruturados para gestão e prestação de contas.
      \end{itemize}
    \item \textbf{Para a instituição}:
      \begin{itemize}
        \item Passo concreto em direção à Gestão 4.0.
        \item Plataforma replicável para outros cursos e campi.
      \end{itemize}
  \end{itemize}
\end{frame}

\begin{frame}{Próximos Passos}
  \begin{itemize}
    \item Expansão do FasiTech para outros formulários e fluxos institucionais.
    \item Evolução do módulo de IA (novos documentos, novos agentes especializados).
    \item Monitoramento de indicadores de uso e tempo de atendimento.
    \item Compartilhamento da experiência com outras unidades da UFPA.
  \end{itemize}
\end{frame}

% =========================
\section{Encerramento}

\begin{frame}{Encerramento}
  \begin{center}
    \Large Obrigado pela atenção!
  \end{center}
  \vfill
  \begin{center}
    \small
    Contato: \href{mailto:eltonss@ufpa.br}{eltonss@ufpa.br}\\[0.2cm]
    \href{https://github.com/eltonsarmanho}{\faGithub\ eltonsarmanho} \quad
    \href{https://www.linkedin.com/in/elton-sarmanho-836553185/}{\faLinkedinSquare\ LinkedIn}
  \end{center}
\end{frame}

\end{document}